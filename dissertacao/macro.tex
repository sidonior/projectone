%% ESte arquivo deve conter os pacotes necess\'arios para a compila\c{c}\~ao 
%% do seu trabalho e o conjunto de todos os novos comando que voc\^e queira 
%% criar. N\~AO ALTERAR O NOME DESTE ARQUIVO.

% pacotes
\usepackage{amssymb} %% Simbolos matem\'aticos
\usepackage[normalem]{ulem} %% para usar o comando de strikeout (\sout)
\usepackage{color} %% Permite usar texto colorido
\usepackage{pdfpages} %% Permite inserir p\'aginas pdf no arquivo latex
\usepackage{enumerate} %% Permite customizar o contador do enviroment enumerate



%criando novas cores
\definecolor{amethyst}{rgb}{0.6, 0.4, 0.8}
\definecolor{ao}{rgb}{0.0, 0.5, 0.0}
\definecolor{tangerine}{rgb}{1.0, 0.6, 0.4}

% Criando comandos de correcao e comentarios
\newcommand{\rem}[1]{{\color{red} \sout{#1}}} % remover texto
\newcommand{\new}[1]{{\color{blue} #1}} % Incluir texto
\newcommand{\com}[1]{{\color{ao} #1}} % Incluir coment\'ario


% conjuntos
\newcommand{\R}{\mathbb{R}}
\newcommand{\C}{\mathbb{C}}
\newcommand{\Z}{\mathbb{Z}}
\newcommand{\N}{\mathbb{N}}


% trigonometria
\newcommand{\sech}{\textrm{sech}}
\newcommand{\csch}{\textrm{csch}}


