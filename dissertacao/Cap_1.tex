%% Todo cap\'itulo deve come\c{c}\~ao com o comando abaixo o qual deve conter o 
%% t\'itulo do mesmo.
\chapter{Introdução}


Ultimamente, como a industria enfrenta áreas geologicamente mais complexas, os métodos de migração baseados na extrapolação dos campos de onda estão sendo aplicados \citep{claerbout_1971, symes_2008}. Mas, isso se tornou possível, devido às técnicas de aquisição que proporcionam uma melhor iluminação da subsuperfície e os recursos computacionais de alta tecnologia. Esses métodos de imageamento da subsuperfície requer modelos cada vez mais refinados, com isso, uma nova ferramenta de construção de modelos baseada na equação da onda foi estudada, na qual chama-se inversão de onda completa (FWI) \citep{fichtner_2011, virieux_overview_2009, tarantola_2005, tarantola_linearized_1984}. Esse estudo está cada vez mais se tornando um elemento estabelecido do fluxo de processamento sísmicos nos estágios de exploração e monitoramento da produção de óleo e gás por fornecer imagens de alta resolução ajudando, principalmente, na interpretação da subsuperfície estudada. \\

Na inversão sísmica, tem-se como objetivo recuperar o modelo da subsuperfície de modo ``exato''. Mas, a inversão da forma de onda completa (FWI) convencional possui uma complexidade por ser um problema não-linear com muitos minímos locais \citep{bunks_1995}, encontrados na função objetivo e, assim, dificulta-se técnicas de otimização iterativa de trabalhar efetivamente, o qual contribui para uma má recuperação de um bom modelo de velocidade. A presença desses mínimos locais  dificulta a ação dos métodos baseados em gradientes de encontrar uma solução global, deste modo, obtém-se um péssimo modelo de velocidade estimado devido à  má escolha do modelo inicial, o que gera grandes erros entre os dados observados e modelados. No entanto, várias abordagens foram desenvolvidas para reduzir o número de mínimos locais na função objetivo.   \\
 
A inversão completa da onda (FWI) tem como idealização a abordagem da otimização, na qual consiste na mínimização do funcional que realiza o ajuste entre os dados observados e modelados, esse funcional chama-se função objetivo. Mas, para utilizar técnicas de otimização, além da função objetivo, há também  uma necessidade do cálculo do gradiente (direção de máximo crescimento da função objetivo) a partir dos métodos dos estados adjuntos \citep{plessix_2006, fichtner_2011, tarantola_2005}.  
Esses métodos de otimização solucionam problemas para encontrar um valor mínimo ou máximo de uma função não-linear, que podem ser determinísticos ou probabilísticos para encontrar uma melhor solução para o problema proposto. Na literatura, há vários métodos que realizam a minimização, entre os mais utilizados são os métodos do gradiente conjugado (CG), método do gradiente, método de Newton e quase-Newton desenvolvidas nas obras públicadas por \citep{nocedal_2006, nocedal_1980,al-baali_broydens_2013, polak_1965,hestenes_1952,fletcher_1964}. \\

O FWI tem potencial para se tornar a principal ferramenta de construção de modelos da subsuperfı́cie de alta resolução. Porém, é bem conhecido a sua sensibilidade à escolha do modelo inicial, devido à falta de informação de baixo comprimento de onda caracterı́stico dos levantamentos sı́smicos de reflexão. A proposta parte de estudo prévio, no qual é proposto a decomposição dos núcleos de sensibilidade do campo de onda acústico com respeito aos parâmetros do modelo em componentes de fundo e singular. Dessa maneira, as estimativas das perturbações para ambas componentes, de fundo e,principalmente, singular, são pré-determinadas a partir de subnúcleos, de ondas multiplamente espalhadas, de interesse para o cálculo do gradiente, que contém informações de baixa frequência \citep{macedo_2014}. O objetivo desta proposta é combinar os usos, de maneira alternada, dos núcleos de sensibilidade convencional e  dos subnúcleos baseado em espalhamento no processo de inversão, buscando, assim, se beneficiar da autoiluminação complementar propiciada pelos espalhadores existentes no próprio meio.
%Para encontrar uma melhor metodologia para avaliar o gradiente da função objetivo, a fim de definir a direção descendente no processamento de minimização, uma método de decomposição foi criado a partir da decomposição dos núcleos de sensibilidade, esse núcleos são usados para avaliar o gradiente \citep{symes_2008,troltzsch_2010}, e assim, definir uma direção de descida que faça o funcional diminuir \citep{tarantola_2005,fichtner_2011,tarantola_1984}. Então, essa metodologia pressupõem que a decomposição dos núcleos de sensibilidade têm diferentes níveis de iterações entre as componentes do modelo (de fundo e singular) que permite a escolha do mais úteis, dependendo do problema estudado. Essa decomposição foi criada por \citep{macedo_2014} para analisar a interação do campo de onda da componente de fundo e o campo de onda da componente singular em cada parte do núcleo decomposto. Precisamente, essa decomposição foi uma forma de avaliação para encontrar uma melhor direção, consequentemente, um função objetivo apresentando poucos mínimos locais, com isso, linearizando o problema por conta do gradiente desse funcional tenha uma melhor direção de descida para encontrar uma solução global. \\









\section{Objetivo do projeto}
O objetivo central deste projeto é testar a efetividade da decomposição proposta em processo completo de inversão, adicionando as contribuições dos subnúcleos de interesse, logo, aproveitam-se as melhores estimativas da direção para o processo inversão. O caminho mais natural possível é combinar os usos, de maneira alternada, os núcleos de sensibilidade convencional baseados no espalhamento único com os subnúcleos baseados no multiespalhamento proposto por \citet{macedo_2014}, no processo de inversão, busca-se, assim, se beneficiar da autoiluminação complementar propiciada pelos espalhadores existentes no próprio meio. Nesta primeira etapa do projeto planejamos definir um número restrito de modelos 2D
de complexidade estrutural variada, na qual separamos em dois modelos, de fundo obtido a partir da suavização do modelo real e o singular ou perturbação do modelo que será obtido a partir de três estratégias, uma delas é realizar o imageamento utilizando o RTM (``Reverse Time Migration''), outra obtida através da primeira atualização da inversão, que será a estimativa do gradiente ou perturbação do modelo e, por fim, a diferença entre o modelo real com o modelo suavizado. A segunda etapa trata-se de efetivar a utilização da nova metodologia de inversão alternada, com a seleção pré-determinada das contribuições dos subnúcleos obtidas nos resultados do trabalho produzido por \citet{macedo_2014}, em suma, será realizado apenas para poucas atualizações, a fim de realizar testes de acurácia  dos códigos modelados. E por fim, realizar o processo de inversão completa da onda, de modo alternado, para determinar as melhores soluções para os problemas propostos.